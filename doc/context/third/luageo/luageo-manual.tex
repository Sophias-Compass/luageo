% !TEX useAlternatePath
% !TEX useConTeXtSyncParser
%
%D \module
%D   [       file=t-luageo,
%D        version=2023.12.16,
%D          title=\CONTEXT\ Extra Modules,
%D       subtitle=\METAPOST\ mp-geo rewrite,
%D         author=Gavin Polhemus,
%D           date=\currentdate,
%D      copyright={Gavin Polhemus}]

% Data source: GLOBE Binaries DECODING : World Public Domain Dbase : F.Pospeschil, A.Rivera (1999)
%  ftp://ftp.blm.gov/pub/gis/wdbprg.zip
% According to https://ctan.math.washington.edu/tex-archive/graphics/pstricks/contrib/pst-geo/doc/pst-geo-doc.pdf

%\usepath[../../../../tex/context/third/luageo,../../../../tex/context/third/luageo/data] % Remove when the module is in the tree.

\usemodule[luageo]

%\usetypescriptfile[libertinus]
%\setupbodyfont [libertinus,11pt]%

%\enabletrackers[metapost*]
%\enabletrackers[metapost.scripts]
%\enabletrackers[metapost.lua]


\starttext

\starttitle[title=luageo]

I should write some documentation.

Drawn with the luageo module:
\blank[big]

\startMPcode
   GlobeDiameter := 10cm ;
   fill fullcircle scaled GlobeDiameter withcolor .9white ;    % Fill a circle with the water color.
   fill globe(40, -108) scaled .5GlobeDiameter withcolor .75white ;% Draw the land.
   draw fullcircle scaled GlobeDiameter withcolor black ;      % Add a border, if you want.
\stopMPcode

The globe below uses a buffer so that it is not redrawn with every typeset.
\startbuffer[demo]
\usemodule [luageo]
\startMPpage
    GlobeDiameter := 6cm ;
    fill fullcircle
        scaled GlobeDiameter
        withcolor .7white + .3blue;
    path p ; p := globe(23, 0)
        scaled .5GlobeDiameter
    ;
    draw p
        withpen pencircle scaled .2mm
        withcolor blue
        withtransparency (1,.5) ;
    fill p
        withcolor darkgreen
        withtransparency (1,.5) ;
    draw fullcircle
        scaled GlobeDiameter
        withcolor black ;
\stopMPpage
\stopbuffer

\typesetbuffer[demo]


\stoptitle

\stoptext
