% !TEX useAlternatePath
% !TEX useConTeXtSyncParser
%
%D \module
%D   [       file=t-luageo,
%D        version=2023.12.16,
%D          title=\CONTEXT\ Extra Modules,
%D       subtitle=\METAPOST\ mp-geo rewrite,
%D         author=Gavin Polhemus,
%D           date=\currentdate,
%D      copyright={Gavin Polhemus}]

% Data source: GLOBE Binaries DECODING : World Public Domain Dbase : F.Pospeschil, A.Rivera (1999)
%  ftp://ftp.blm.gov/pub/gis/wdbprg.zip
% According to https://ctan.math.washington.edu/tex-archive/graphics/pstricks/contrib/pst-geo/doc/pst-geo-doc.pdf

\usemodule[luageo]

%\usetypescriptfile[libertinus]
%\setupbodyfont [libertinus,11pt]%

%\enabletrackers[metapost*]
%\enabletrackers[metapost.scripts]
%\enabletrackers[metapost.lua]

\setuppagenumbering[location=footer]

% \nopdfcompression

\setupinteraction[state=start]
%\setupcolors[state=start]
%\usemodule[luageo][color=rothko] % example of a module parameter

%\usetypescriptfile[libertinus]
%\setupbodyfont [libertinus,11pt]%


\starttext

\startalignment[center]
  {\tfd Luageo}
  \blank[medium]
  by Gavin Polhemus
\stopalignment

\blank[big]

\startbuffer[demo]
\usemodule [luageo]
\startMPpage
  GlobeDiameter := 10cm ;
  path p ; p := globe(40, -108) scaled GlobeDiameter ;
  % Fill a circle with the water color
  fill fullcircle
    scaled GlobeDiameter
    withcolor .7white + .3blue ;
  % Fill the land
  fill p
    withcolor darkgreen
    withtransparency (1,.5) ;
  % Draw the borders.
  draw p
    withpen pencircle scaled .2mm
    withcolor blue
    withtransparency (1,.5) ;
  % Draw a horizon circle
  draw fullcircle
    scaled GlobeDiameter
    withcolor black ;
  % Add other picture elements
  label(textext("*"),origin) ;
\stopMPpage
\stopbuffer

\startalignment[center]
\dontleavehmode
\typesetbuffer[demo]
\stopalignment


\startsubject[title=Introduction]
Luageo is simple \ConTeXt\ module for drawing globes. A globe can be drawn with any location centered using the MetaPost macro \type{globe( , )}. The two arguments are the latitude and longitude of the center location. The macro returns a single MetaPost path consisting of all visible country and island boundaries. This path can be drawn or filled. For example, the MetaPost commands

\startalignment[center]
\type{draw globe(21,0) scaled 3cm ;}
\stopalignment
and
\startalignment[center]
\type{fill globe(21,0) scaled 3cm ;}
\stopalignment

produce

\startalignment[center]
\dontleavehmode
\startMPcode
   draw globe(21,0) scaled 3cm xshifted -3cm ;
   label(textext("and"),origin) ;
   fill globe(21,0) scaled 3cm xshifted 3cm ;
\stopMPcode
\stopalignment


To draw a complete globe either start by filling a circle with the ocean color or finish by drawing a horizon circle. You can do both, as in the complete example code below. This code produces the large globe on this manual's first page.

\typebuffer[demo]

The code above was placed in a buffer so the globe is not redrawn with every typeset.

%\startbuffer[example]
%\usemodule [luageo]
%\startMPpage
%   GlobeDiameter := 5cm ;
%   % Fill a circle with the water color.
%   fill fullcircle scaled GlobeDiameter withcolor .9white ;
%   % Draw the land.
%   fill globe(40, -108) scaled GlobeDiameter withcolor .75white ;
%   % Add a border, if you want.
%   draw fullcircle scaled GlobeDiameter withcolor black ;
%\stopMPpage
%\stopbuffer
%
%\typebuffer[example]
%
%\startalignment[center]
%\dontleavehmode
%\typesetbuffer[example]
%\stopalignment

\stopsubject

\startsubject[title={Limitations and experiments}]

Currently, Luageo only draws complete globes, not other maps. The path always includes country boundaries but no other features. (Large lakes are visible if they are on country boundaries.)
You cannot fill the countries with different colors. You cannot draw only the coasts.
I do not know when the country boundaries were last updated. They do show a unified Germany, so the map data is not too ancient.

The level of detail in Luageo globes is excessive, resulting in very large file sizes. For example, the globe at the beginning of this manual is 2.5MB. The size can be reduced using an experimental command \type{globetrot( , , )}.
The third argument is a step size, telling Luageo to skip points when calculating the path. (For example, a step size of 2 will skip every other point.) The globe below uses a step size of 20 and is only 73KB.

\startbuffer[exampletrot]
\usemodule [luageo]
\startMPpage
   GlobeDiameter := 10cm ;
   % Fill a circle with the water color.
   fill fullcircle scaled GlobeDiameter withcolor blue ;
   % Draw the land.
   fill globetrot(23, 0, 20) scaled GlobeDiameter withcolor green ;
   % Add a border, if you want.
   draw fullcircle scaled GlobeDiameter withcolor black ;
\stopMPpage
\stopbuffer

\typebuffer[exampletrot]

\startalignment[center]
\dontleavehmode
\typesetbuffer[exampletrot]
\stopalignment

However, notice the spotty country boundaries. This happens because the boundaries between countries are are included in the path twice, once for each country.
If the step mechanism selects different points for each country along the shared border, the two borders will not match. The resulting gaps are visible.

The \type{globetrot} command may be depricated if a better solution is found.


%\dostepwiserecurse{0}{360}{5}{
%\startMPpage[offset=1dk]
%    GlobeDiameter := 8cm ;
%    path p ; p := globe(30,#1)
%%    path p ; p := globetrot(30,#1,7)
%        scaled .5GlobeDiameter
%    ;
%    fill fullcircle
%        scaled GlobeDiameter
%        withcolor blue
%        withtransparency (1,.3)
%    ;
%    draw p
%    withcurvature 0
%         withpen pencircle scaled .2mm
%%        withpen pencircle scaled .01mm
%    ;
%    fill p
%        withcolor darkgreen
%        withtransparency (1,.8)
%    ;
%%    draw fullcircle
%%        scaled GlobeDiameter
%%    ;

%\enabletrackers[metapost*]
%\enabletrackers[metapost.scripts]
%\enabletrackers[metapost.lua]

\stopsubject

\startsubject[title={Contribute to Luageo}]

Luageo is a small module, containing only a few hundred lines of code (in addition to the data files). Most of the code is written in Lua, with some help from MetaPost and \ConTeXt. If you are interested in integrating Lua and MetaPost for graphics in \ConTeXt, Luageo might be an interesting example for you.
Visit \goto{Luageo on GitHub}[url(https://github.com/Sophias-Compass/Luageo)] to improve Luageo or to use it as a starting point for your own \ConTeXt+MetaPost+Lua project.

There are several obvious opportunities for improving Luageo:

\startitemize
  \item More \ConTeXt-like behavior. For example, adding a \type{key=value} interface.
  \item More globe views and map projections.
  \item Better map data. For example, \goto{Natural Earth}[url(https://www.naturalearthdata.com)] has less detailed data appropriate for Luageo's small globes, as well as more detailed data for zoomed-in maps of countries or regions. For local maps, \goto{OpenStreetMap}[url(https://www.openstreetmap.org)] might be a good source.
\stopitemize

You can also find some of my other projects on GitHub. I created Luageo to help with diagrams for a physics textbook. Eventually, I hope to write other modules related to the textbook project.
I would love to have more collaborators. You don't have to know any physics!

\startsubject[title={Acknowledgements}]

The Luageo module was inspired by the MetaPost extension \goto{mp-geo}[url(https://melusine.eu.org/syracuse/G/git/?p=mp-geo.git;a=summary)] by Christophe Poulain. No code from mp-geo is used in Luageo, but the map data came from the mp-geo extension.

The map data is adapted from the \goto{CIA World DataBank II}[url(http://www.evl.uic.edu/pape/data/WDB/)]. At some point, this data was available at GLOBE Binaries DECODING : World Public Domain Dbase : F.Pospeschil, A.Rivera (1999), ftp://ftp.blm.gov/pub/gis/wdbprg.zip (according to the \goto{pst-geo maual}[url(https://ctan.math.washington.edu/tex-archive/graphics/pstricks/contrib/pst-geo/doc/pst-geo-doc.pdf)], p.4).

Hans Hagen offered many helpful suggestions, some improvements to the code, and general encouragement. I received helpful guidance from Henning Hraban Ramm as I prepared the module for sharing. Thanks!

\stopsubject

\stoptext
